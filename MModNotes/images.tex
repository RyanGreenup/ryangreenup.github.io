\batchmode
\documentclass{report}
\RequirePackage{ifthen}


\usepackage{diffpreamble}


\usepackage{xcolor}

\usepackage[]{inputenc}



\makeatletter
\AtBeginDocument{\makeatletter
\input /home/ryan/Dropbox/Studies/MathModelling/Notes/latex_working/00_master.aux
\makeatother
}
\AtBeginDocument{\makeatletter
\input /home/ryan/Dropbox/Studies/MathModelling/Notes/latex_working/01_SepTex.aux
\makeatother
}

\makeatletter
\count@=\the\catcode`\_ \catcode`\_=8 
\newenvironment{tex2html_wrap}{}{}%
\catcode`\<=12\catcode`\_=\count@
\newcommand{\providedcommand}[1]{\expandafter\providecommand\csname #1\endcsname}%
\newcommand{\renewedcommand}[1]{\expandafter\providecommand\csname #1\endcsname{}%
  \expandafter\renewcommand\csname #1\endcsname}%
\newcommand{\newedenvironment}[1]{\newenvironment{#1}{}{}\renewenvironment{#1}}%
\let\newedcommand\renewedcommand
\let\renewedenvironment\newedenvironment
\makeatother
\let\mathon=$
\let\mathoff=$
\ifx\AtBeginDocument\undefined \newcommand{\AtBeginDocument}[1]{}\fi
\newbox\sizebox
\setlength{\hoffset}{0pt}\setlength{\voffset}{0pt}
\addtolength{\textheight}{\footskip}\setlength{\footskip}{0pt}
\addtolength{\textheight}{\topmargin}\setlength{\topmargin}{0pt}
\addtolength{\textheight}{\headheight}\setlength{\headheight}{0pt}
\addtolength{\textheight}{\headsep}\setlength{\headsep}{0pt}
\setlength{\textwidth}{349pt}
\newwrite\lthtmlwrite
\makeatletter
\let\realnormalsize=\normalsize
\global\topskip=2sp
\def\preveqno{}\let\real@float=\@float \let\realend@float=\end@float
\def\@float{\let\@savefreelist\@freelist\real@float}
\def\liih@math{\ifmmode$\else\bad@math\fi}
\def\end@float{\realend@float\global\let\@freelist\@savefreelist}
\let\real@dbflt=\@dbflt \let\end@dblfloat=\end@float
\let\@largefloatcheck=\relax
\let\if@boxedmulticols=\iftrue
\def\@dbflt{\let\@savefreelist\@freelist\real@dbflt}
\def\adjustnormalsize{\def\normalsize{\mathsurround=0pt \realnormalsize
 \parindent=0pt\abovedisplayskip=0pt\belowdisplayskip=0pt}%
 \def\phantompar{\csname par\endcsname}\normalsize}%
\def\lthtmltypeout#1{{\let\protect\string \immediate\write\lthtmlwrite{#1}}}%
\usepackage[tightpage,active]{preview}
\newbox\lthtmlPageBox
\newdimen\lthtmlCropMarkHeight
\newdimen\lthtmlCropMarkDepth
\long\def\lthtmlTightVBox#1#2{%
    \setbox\lthtmlPageBox\vbox{\hbox{\catcode`\_=8 #2}}%
    \lthtmlCropMarkHeight=\ht\lthtmlPageBox \advance \lthtmlCropMarkHeight 6pt
    \lthtmlCropMarkDepth=\dp\lthtmlPageBox
    \lthtmltypeout{^^J:#1:lthtmlCropMarkHeight:=\the\lthtmlCropMarkHeight}%
    \lthtmltypeout{^^J:#1:lthtmlCropMarkDepth:=\the\lthtmlCropMarkDepth:1ex:=\the \dimexpr 1ex}%
    \begin{preview}\copy\lthtmlPageBox\end{preview}}}%
\long\def\lthtmlTightFBox#1#2{%
    \adjustnormalsize\setbox\lthtmlPageBox=\vbox\bgroup %
    \let\ifinner=\iffalse \let\)\liih@math %
    {\catcode`\_=8 #2}%
    \@next\next\@currlist{}{\def\next{\voidb@x}}%
    \expandafter\box\next\egroup %
    \lthtmlCropMarkHeight=\ht\lthtmlPageBox \advance \lthtmlCropMarkHeight 6pt
    \lthtmlCropMarkDepth=\dp\lthtmlPageBox
    \lthtmltypeout{^^J:#1:lthtmlCropMarkHeight:=\the\lthtmlCropMarkHeight}%
    \lthtmltypeout{^^J:#1:lthtmlCropMarkDepth:=\the\lthtmlCropMarkDepth:1ex:=\the \dimexpr 1ex}%
    \begin{preview}\copy\lthtmlPageBox\end{preview}}%
    \long\def\lthtmlinlinemathA#1#2\lthtmlindisplaymathZ{\lthtmlTightVBox{#1}{#2}}
    \def\lthtmlinlineA#1#2\lthtmlinlineZ{\lthtmlTightVBox{#1}{#2}}
    \long\def\lthtmldisplayA#1#2\lthtmldisplayZ{\lthtmlTightVBox{#1}{#2}}
    \long\def\lthtmlinlinemathA#1#2\lthtmlindisplaymathZ{\lthtmlTightVBox{#1}{#2}}
    \def\lthtmlinlineA#1#2\lthtmlinlineZ{\lthtmlTightVBox{#1}{#2}}
    \long\def\lthtmldisplayA#1#2\lthtmldisplayZ{\lthtmlTightVBox{#1}{#2}}
    \long\def\lthtmldisplayB#1#2\lthtmldisplayZ{\\edef\preveqno{(\theequation)}%
        \lthtmlTightVBox{#1}{\let\@eqnnum\relax#2}}
    \long\def\lthtmlfigureA#1#2\lthtmlfigureZ{\let\@savefreelist\@freelist
        \lthtmlTightFBox{#1}{#2}\global\let\@freelist\@savefreelist}
    \long\def\lthtmlpictureA#1#2\lthtmlpictureZ{\let\@savefreelist\@freelist
        \lthtmlTightVBox{#1}{#2}\global\let\@freelist\@savefreelist}
\def\lthtmlcheckvsize{\ifdim\ht\sizebox<\vsize 
  \ifdim\wd\sizebox<\hsize\expandafter\hfill\fi \expandafter\vfill
  \else\expandafter\vss\fi}%
\providecommand{\selectlanguage}[1]{}%
\makeatletter \tracingstats = 1 


\begin{document}
\pagestyle{empty}\thispagestyle{empty}\lthtmltypeout{}%
\lthtmltypeout{latex2htmlLength hsize=\the\hsize}\lthtmltypeout{}%
\lthtmltypeout{latex2htmlLength vsize=\the\vsize}\lthtmltypeout{}%
\lthtmltypeout{latex2htmlLength hoffset=\the\hoffset}\lthtmltypeout{}%
\lthtmltypeout{latex2htmlLength voffset=\the\voffset}\lthtmltypeout{}%
\lthtmltypeout{latex2htmlLength topmargin=\the\topmargin}\lthtmltypeout{}%
\lthtmltypeout{latex2htmlLength topskip=\the\topskip}\lthtmltypeout{}%
\lthtmltypeout{latex2htmlLength headheight=\the\headheight}\lthtmltypeout{}%
\lthtmltypeout{latex2htmlLength headsep=\the\headsep}\lthtmltypeout{}%
\lthtmltypeout{latex2htmlLength parskip=\the\parskip}\lthtmltypeout{}%
\lthtmltypeout{latex2htmlLength oddsidemargin=\the\oddsidemargin}\lthtmltypeout{}%
\makeatletter
\if@twoside\lthtmltypeout{latex2htmlLength evensidemargin=\the\evensidemargin}%
\else\lthtmltypeout{latex2htmlLength evensidemargin=\the\oddsidemargin}\fi%
\lthtmltypeout{}%
\makeatother
\setcounter{page}{1}
\onecolumn

% !!! IMAGES START HERE !!!

\stepcounter{part}
\stepcounter{chapter}
\stepcounter{section}
\stepcounter{subsection}
\stepcounter{subsubsection}
{\newpage\clearpage
\lthtmlfigureA{align16}%
\begin{align}
\frac{\operatorname{d} }{\operatorname{d} x}\left( u\cdot v \right)&= \frac{\operatorname{d} u}{\operatorname{d} v}\cdot v + u \cdot \frac{\operatorname{d} v}{\operatorname{d} x}
\\
\ \notag \\
\frac{\operatorname{d} }{\operatorname{d} x}\left( f\left( x \right)\cdot g\left( x \right) \right)&= f'\left( x \right)\cdot g\left( x \right)+ f\left( x \right)\cdot g'\left( x \right)
\end{align}%
\lthtmlfigureZ
\lthtmlcheckvsize\clearpage}

\stepcounter{subsubsection}
{\newpage\clearpage
\lthtmlfigureA{align30}%
\begin{align}
  \frac{\operatorname{d}y }{\operatorname{d} x} &= \frac{\operatorname{d} y}{\operatorname{d} u} \cdot \frac{\operatorname{d} u}{\operatorname{d} x}\\
\ \notag \\
  \frac{\operatorname{d} }{\operatorname{d} x}\left[ f\left( g\left( x \right) \right) \right]&= f'\left( g\left( x \right) \right)\cdot g\left( x \right) \\
  \ \notag  
\end{align}%
\lthtmlfigureZ
\lthtmlcheckvsize\clearpage}

\stepcounter{subsection}
\stepcounter{subsubsection}
{\newpage\clearpage
\lthtmlfigureA{align44}%
\begin{align}
\begin{matrix}
u &= g(x) & \quad F(x): \enspace  F'(x) = f(x) = y \\
\frac{du}{dx} &= g'(x)
\end{matrix}
\end{align}%
\lthtmlfigureZ
\lthtmlcheckvsize\clearpage}

{\newpage\clearpage
\lthtmlfigureA{align50}%
\begin{align}
\frac{\operatorname{d} }{\operatorname{d} x}\left[ F'\left( u \right) \right]&= F'\left( g\left( x \right) \right)\cdot g'\left( x \right)\notag \\
&= f\left( g\left( x \right) \right)\cdot g'\left( x \right)\notag \\
 \implies  f\left( g\left( x \right) \right)\cdot g'\left( x \right)&= \frac{\operatorname{d} }{\operatorname{d} x}\left[ F\left( u \right) \right]\notag \\
 f\left( g\left( x \right) \right)\cdot g'\left( x \right)&= \frac{\operatorname{d} }{\operatorname{d} x}\left[ F\left( u \right) +C \right]\notag \\
\end{align}%
\lthtmlfigureZ
\lthtmlcheckvsize\clearpage}

{\newpage\clearpage
\lthtmlfigureA{align58}%
\begin{align}
\int^{}_{} f\left( g\left( x \right) \right)\cdot g'\left( x \right)  \operatorname{d}x &= \int^{}_{} \frac{\operatorname{d} }{\operatorname{d} x}\left[ F\left( u \right)+ C \right]  \operatorname{d}x \notag \\
&= F\left(u  \right) + C\notag \\
&= \int^{}_{} f\left( u \right)  \operatorname{d}u\notag  
\end{align}%
\lthtmlfigureZ
\lthtmlcheckvsize\clearpage}

{\newpage\clearpage
\lthtmlfigureA{align71}%
\begin{align}
\int^{}_{} f\left( g\left( x \right) \right)\cdot g'\left( x \right)  \operatorname{d}x &= \int^{}_{} f\left( u \right)  \operatorname{d}u
\\
\int^{}_{} f\left( u \right)\cdot \frac{\operatorname{d}u }{\operatorname{d} x}  \operatorname{d}x&= \int^{}_{} f\left( u \right)  \operatorname{d}u
\end{align}%
\lthtmlfigureZ
\lthtmlcheckvsize\clearpage}

\stepcounter{subsubsection}
{\newpage\clearpage
\lthtmlinlinemathA{tex2html_wrap_inline387}%
$u= f\left( x \right)$%
\lthtmlindisplaymathZ
\lthtmlcheckvsize\clearpage}

{\newpage\clearpage
\lthtmlinlinemathA{tex2html_wrap_inline389}%
$\frac{\operatorname{d} }{\operatorname{d} x}\left[ \sin{\left( x \right)} \right]= \cos{\left( x \right)}$%
\lthtmlindisplaymathZ
\lthtmlcheckvsize\clearpage}

{\newpage\clearpage
\lthtmlinlinemathA{tex2html_wrap_inline391}%
$\operatorname{d} v= g'\left( x \right) \operatorname{d} x$%
\lthtmlindisplaymathZ
\lthtmlcheckvsize\clearpage}

{\newpage\clearpage
\lthtmlinlinemathA{tex2html_wrap_inline393}%
$v$%
\lthtmlindisplaymathZ
\lthtmlcheckvsize\clearpage}

{\newpage\clearpage
\lthtmlinlinemathA{tex2html_wrap_inline395}%

% latex2html id marker 395
$\left( \ref{prodruledefleib} \right)$%
\lthtmlindisplaymathZ
\lthtmlcheckvsize\clearpage}

{\newpage\clearpage
\lthtmlfigureA{align101}%
\begin{align}
\frac{\operatorname{d} }{\operatorname{d} x}\left[ f\left( x \right)\cdot g\left( x \right) \right]&= f'\left( x \right)\cdot g\left( x \right)+ f\left( x \right)\cdot g'\left( x \right)\notag \\
 \end{align}%
\lthtmlfigureZ
\lthtmlcheckvsize\clearpage}

{\newpage\clearpage
\lthtmlfigureA{align105}%
\begin{align}
\begin{matrix}
  &u = f\left( x \right) &&v = g\left( x \right)\\
  &\frac{\operatorname{d}u }{\operatorname{d} x}= f'\left( x \right) && \frac{\operatorname{d}v }{\operatorname{d} x} = g'\left( x \right)
\end{matrix}
\end{align}%
\lthtmlfigureZ
\lthtmlcheckvsize\clearpage}

{\newpage\clearpage
\lthtmlfigureA{align113}%
\begin{align}
  \int^{}_{} \left( \frac{\operatorname{d}u }{\operatorname{d} x}\cdot v + u\cdot \frac{\operatorname{d}v }{\operatorname{d} x} \right)  \operatorname{d}x &= u\cdot v\notag \\
  \ \notag \\
\int^{}_{} \left( v\cdot \frac{\operatorname{d}u }{\operatorname{d} x} \right)  \operatorname{d}x +  \int^{}_{} \left( u\cdot \frac{\operatorname{d}v }{\operatorname{d} x} \right)  \operatorname{d}x&= u\cdot v  \notag 
\end{align}%
\lthtmlfigureZ
\lthtmlcheckvsize\clearpage}

{\newpage\clearpage
\lthtmlinlinemathA{tex2html_wrap_inline379}%

% latex2html id marker 379
\(\left( \ref{ibysubl} \right)\)%
\lthtmlindisplaymathZ
\lthtmlcheckvsize\clearpage}

{\newpage\clearpage
\lthtmlfigureA{align133}%
\begin{align}
\int^{}_{} v  \operatorname{d}u +  \int^{}_{} u  \operatorname{d}v &= u\cdot v  \notag \\
\int^{}_{} u  \operatorname{d}v &= u\cdot v - \int^{}_{} v  \operatorname{d}u
\end{align}%
\lthtmlfigureZ
\lthtmlcheckvsize\clearpage}

\stepcounter{paragraph}
{\newpage\clearpage
\lthtmlinlinemathA{tex2html_wrap_inline380}%
\(\int^{}_{} \left[\enspace  \right]  \operatorname{d}x \)%
\lthtmlindisplaymathZ
\lthtmlcheckvsize\clearpage}

{\newpage\clearpage
\lthtmlinlinemathA{tex2html_wrap_inline381}%
\(\left[ f\left( u \right)\cdot \frac{\operatorname{d}u }{\operatorname{d} x}\right] = \left[ f\left( g\left( x \right) \right)\cdot g'\left( x \right) \right]\)%
\lthtmlindisplaymathZ
\lthtmlcheckvsize\clearpage}

{\newpage\clearpage
\lthtmlinlinemathA{tex2html_wrap_inline382}%
\(\left[ f\left( x \right)\cdot \frac{\operatorname{d}u }{\operatorname{d} x} \right]= \left[ f\left( x \right)\cdot g'\left( x \right) \right]\)%
\lthtmlindisplaymathZ
\lthtmlcheckvsize\clearpage}

\stepcounter{subsection}
{\newpage\clearpage
\lthtmlinlinemathA{tex2html_wrap_inline397}%
$\operatorname{d} y$%
\lthtmlindisplaymathZ
\lthtmlcheckvsize\clearpage}

{\newpage\clearpage
\lthtmlinlinemathA{tex2html_wrap_inline399}%
$\operatorname{d} x$%
\lthtmlindisplaymathZ
\lthtmlcheckvsize\clearpage}

{\newpage\clearpage
\lthtmlinlinemathA{tex2html_wrap_inline401}%
$\frac{\operatorname{d}y }{\operatorname{d} x}$%
\lthtmlindisplaymathZ
\lthtmlcheckvsize\clearpage}

{\newpage\clearpage
\lthtmlinlinemathA{tex2html_wrap_inline403}%
$\frac{\partial u }{\partial x}$%
\lthtmlindisplaymathZ
\lthtmlcheckvsize\clearpage}

\stepcounter{paragraph}
\stepcounter{paragraph}
\stepcounter{paragraph}
{\newpage\clearpage
\lthtmlfigureA{align180}%
\begin{align}
  \sum^{n}_{0}   \left[ a_0\left( x \right)\cdot \left( \frac{\operatorname{d}^ny }{\operatorname{d} x^n} \right) \right]
\end{align}%
\lthtmlfigureZ
\lthtmlcheckvsize\clearpage}

\stepcounter{paragraph}
{\newpage\clearpage
\lthtmlinlinemathA{tex2html_wrap_inline405}%
$y$%
\lthtmlindisplaymathZ
\lthtmlcheckvsize\clearpage}

{\newpage\clearpage
\lthtmlinlinemathA{tex2html_wrap_inline407}%
$x$%
\lthtmlindisplaymathZ
\lthtmlcheckvsize\clearpage}

\stepcounter{subsection}
{\newpage\clearpage
\lthtmlfigureA{tcolorbox193}%
\begin{tcolorbox}
\par
A differential equation of the form:
\begin{align}
g\left( y \right)\cdot \frac{\operatorname{d}y }{\operatorname{d} x} = f\left( x \right)
\end{align}
Is a seperable Ordinary Differential Equation and has a solution:
\par
\begin{align}
\int^{}_{} g\left( y \right)  \operatorname{d}y = \int^{}_{} f\left( x \right)  \operatorname{d}x
\end{align}
\end{tcolorbox}%
\lthtmlfigureZ
\lthtmlcheckvsize\clearpage}

\stepcounter{paragraph}
{\newpage\clearpage
\lthtmlfigureA{align211}%
\begin{align}
g\left( y \right)\cdot \frac{\operatorname{d}y }{\operatorname{d} x} &= f\left( x \right)\notag \\
 \implies  \int^{}_{} g\left( y \right)\frac{\operatorname{d}y }{\operatorname{d} x}  \operatorname{d}x &= \int^{}_{} f\left( x \right)  \operatorname{d}x\notag \\
\end{align}%
\lthtmlfigureZ
\lthtmlcheckvsize\clearpage}

{\newpage\clearpage
\lthtmlinlinemathA{tex2html_wrap_inline409}%

% latex2html id marker 409
$\ref{ibysubl}$%
\lthtmlindisplaymathZ
\lthtmlcheckvsize\clearpage}

{\newpage\clearpage
\lthtmlfigureA{align224}%
\begin{align}
\int^{}_{} g\left( y \right)   \operatorname{d}y &= \int^{}_{} f\left( x \right)  \operatorname{d}x  
\end{align}%
\lthtmlfigureZ
\lthtmlcheckvsize\clearpage}

\stepcounter{subsection}
{\newpage\clearpage
\lthtmlinlinemathA{tex2html_wrap_inline411}%
$u$%
\lthtmlindisplaymathZ
\lthtmlcheckvsize\clearpage}

{\newpage\clearpage
\lthtmldisplayA{displaymath413}%
\begin{displaymath}
\frac{\operatorname{d}y }{\operatorname{d} x}= f\left( \frac{x}{y} \right)
\end{displaymath}%
\lthtmldisplayZ
\lthtmlcheckvsize\clearpage}

{\newpage\clearpage
\lthtmlfigureA{align237}%
\begin{align}
  u&= \frac{y}{x}\notag \\
   \implies   y &= u\cdot x\notag \\
    \implies  \frac{\operatorname{d}y }{\operatorname{d} x}&= \frac{\operatorname{d}u }{\operatorname{d} x}\cdot x+ \left( 1 \right)\cdot u \notag
\end{align}%
\lthtmlfigureZ
\lthtmlcheckvsize\clearpage}

{\newpage\clearpage
\lthtmlfigureA{align245}%
\begin{align}
  \frac{\operatorname{d}y }{\operatorname{d} x}&= f\left( \frac{y}{x} \right)\notag \\
  \frac{\operatorname{d}u }{\operatorname{d} x}\cdot x + u&= f\left( u \right)\notag \\
  \frac{\operatorname{d}u }{\operatorname{d} x}\cdot x&= f\left( u \right)- u\notag \\
  \frac{1}{f\left( u \right)- u}\cdot \frac{\operatorname{d}u }{\operatorname{d} x}\cdot x  &= 1\notag \\
  \frac{1}{f\left( u \right)- u }\cdot \frac{\operatorname{d}u }{\operatorname{d} x}&= \int^{}_{} \frac{1}{x}  \operatorname{d}x \notag \\
  \int^{}_{} \frac{1}{f\left( u \right)- u}\cdot \frac{\operatorname{d}u }{\operatorname{d} x}  \operatorname{d}x&= \int^{}_{} \frac{1}{x}   \operatorname{d}x\notag \\
  \int^{}_{} \frac{1}{f\left( u \right)- u}  \operatorname{d}u &= \ln{ \left| x \right| }+ c  
\end{align}%
\lthtmlfigureZ
\lthtmlcheckvsize\clearpage}

{\newpage\clearpage
\lthtmlinlinemathA{tex2html_wrap_inline417}%
$\exists G\left( u \right): G\left( u \right)= \int^{}_{} \frac{1}{f\left( u \right)- u}  \operatorname{d}u $%
\lthtmlindisplaymathZ
\lthtmlcheckvsize\clearpage}

{\newpage\clearpage
\lthtmlfigureA{align291}%
\begin{align}
  G\left( u \right)&= \ln{ \left| x \right| }+ c\notag \\
 G\left( \frac{y}{x} \right)&= \ln{ \left| x \right| }+ c\notag \\
 G\left( \frac{y}{x} \right)+ \ln{ \left| x \right| }+ c &= 0
\end{align}%
\lthtmlfigureZ
\lthtmlcheckvsize\clearpage}

\stepcounter{section}
\stepcounter{subsection}
{\newpage\clearpage
\lthtmlfigureA{alignstar537}%
\begin{align*}
&\sum^{n}_{0}   \left[ a_n\left( x \right)\cdot f^{\left( n \right)} \left( x \right)\right] = g\left( x \right)\\
&\qquad  \qquad {\footnotesize\text{If $g\left( x \right)= 0$\  it is said to be homogenous}}
\end{align*}%
\lthtmlfigureZ
\lthtmlcheckvsize\clearpage}

{\newpage\clearpage
\lthtmlfigureA{align543}%
\begin{align}
  &a_1\left( x \right)\cdot \frac{\operatorname{d}y }{\operatorname{d} x}+ a_0\left( x \right)\cdot y= g\left( x \right)\notag \\
  & \qquad  \qquad    {\footnotesize\text{Where $a\left( x \right)$\  is a function } }
\end{align}%
\lthtmlfigureZ
\lthtmlcheckvsize\clearpage}

{\newpage\clearpage
\lthtmlfigureA{tcolorbox550}%
\begin{tcolorbox}
\par
\textbf{Linear First Order ODE:}
\begin{align}
\frac{\operatorname{d}y }{\operatorname{d} x} + p\left( x \right)\cdot y    = f\left( x \right)
\end{align}
if $f\left( x \right)= 0$\  the equation is said to be homogenous
\end{tcolorbox}%
\lthtmlfigureZ
\lthtmlcheckvsize\clearpage}

{\newpage\clearpage
\lthtmlinlinemathA{tex2html_wrap_inline1001}%
$y_h$%
\lthtmlindisplaymathZ
\lthtmlcheckvsize\clearpage}

{\newpage\clearpage
\lthtmlinlinemathA{tex2html_wrap_inline1003}%
$y_p$%
\lthtmlindisplaymathZ
\lthtmlcheckvsize\clearpage}

{\newpage\clearpage
\lthtmlinlinemathA{tex2html_wrap_inline1005}%
$y_h: \quad \frac{\operatorname{d}y_h }{\operatorname{d} x}+ p\left( x \right)\cdot y_h= 0$%
\lthtmlindisplaymathZ
\lthtmlcheckvsize\clearpage}

{\newpage\clearpage
\lthtmlinlinemathA{tex2html_wrap_inline1007}%
$y_p: \quad \frac{\operatorname{d}y_p }{\operatorname{d} x}+ p\left( x \right)\cdot y_p= f\left( x \right)$%
\lthtmlindisplaymathZ
\lthtmlcheckvsize\clearpage}

{\newpage\clearpage
\lthtmlfigureA{tcolorbox566}%
\begin{tcolorbox}
\par
\begin{itemize}
    \item Rewrite the Equation in the standard form:
      \begin{displaymath}
      \frac{\operatorname{d}y }{\operatorname{d} x}+ p\left( x \right)\cdot y = f \left( x \right)  
      \end{displaymath}
    \item Identify $p\left( x \right)$\  and find the integrating factor: 
      \begin{displaymath}
      e^{\int^{}_{} p\left( x \right)  \operatorname{d}x }
      \end{displaymath} 
      \item Multiply through by the integrating factor:
        \begin{displaymath}
    e^{\int^{}_{} p\left( x \right)  \operatorname{d}x } \left(  \frac{\operatorname{d}y }{\operatorname{d} x}+ p\left( x \right)\cdot y \right) = e^{\int^{}_{} p\left( x \right)  \operatorname{d}x }f \left( x \right)  
        \end{displaymath}
\par
\subitem It may be concluded:
        \ \
\par
\hfill\begin{minipage}{\dimexpr\textwidth-3cm}
        \begin{displaymath}
        \frac{\operatorname{d} }{\operatorname{d} x}\left[ e^{\int^{}_{} p\left( x \right)  \operatorname{d}x \cdot  } \cdot y\right] = e^{\int^{}_{} p\left( x \right)  \operatorname{d}x } \cdot f \left( x \right) 
        \end{displaymath}
        \end{minipage}
        \ \
\par
\item Integrate both sides in order to solve:
  \end{itemize}
\par
\end{tcolorbox}%
\lthtmlfigureZ
\lthtmlcheckvsize\clearpage}

\stepcounter{subsection}
{\newpage\clearpage
\lthtmlfigureA{align596}%
\begin{align}
\frac{\operatorname{d}y }{\operatorname{d} x} +  p\left( x \right)\cdot  y= f\left( x \right)
\end{align}%
\lthtmlfigureZ
\lthtmlcheckvsize\clearpage}

{\newpage\clearpage
\lthtmlfigureA{alignat601}%
\begin{alignat}{2}
  \frac{\operatorname{d}y }{\operatorname{d} x}+ p\left( x \right)\cdot  y&= 0 &\implies  y= y_h
\\
  \frac{\operatorname{d}y }{\operatorname{d} x}+ p\left( x \right)\cdot  y&= f(x) &\implies  y= y_p
\end{alignat}%
\lthtmlfigureZ
\lthtmlcheckvsize\clearpage}

{\newpage\clearpage
\lthtmlfigureA{align610}%
\begin{align}
\frac{\operatorname{d} }{\operatorname{d} x}\left( y_h+ y_p \right)+ p\left( x \right)\cdot  \left( y_h+ y_p \right)&= f\left( x \right)\notag \\
\frac{\operatorname{d}y_h }{\operatorname{d} x}+ \frac{\operatorname{d}y_p }{\operatorname{d} x}+ p\left( x \right)\cdot  y_h + p\left( x \right)\cdot  y_p   &= f\left( x \right)\notag \\
\frac{\operatorname{d}y_h }{\operatorname{d} x}+ p\left( x \right)\cdot  y_h    + \frac{\operatorname{d}y_p }{\operatorname{d} x}+ p\left( x \right)\cdot  y_p  &= f\left( x \right)\notag \\
0 +  f\left( x \right)&= f\left( x \right)
\end{align}%
\lthtmlfigureZ
\lthtmlcheckvsize\clearpage}

{\newpage\clearpage
\lthtmlfigureA{align624}%
\begin{align}
\frac{\operatorname{d}y }{\operatorname{d} x}+ p\left( x \right) \cdot  y &= 0\notag \\
\frac{1}{y}\cdot  \frac{\operatorname{d}y }{\operatorname{d} x}&= - p\left( x \right)\notag \\
\ln{ \left| y \right| }&= \int^{}_{} - p\left( x \right)  \operatorname{d}+ c\notag \\
\left| y \right|&= e^{\int^{- p\left( x \right)x}_{}   \operatorname{d}x }\cdot  e^c
\end{align}%
\lthtmlfigureZ
\lthtmlcheckvsize\clearpage}

{\newpage\clearpage
\lthtmlinlinemathA{tex2html_wrap_inline1023}%
$y>0$%
\lthtmlindisplaymathZ
\lthtmlcheckvsize\clearpage}

{\newpage\clearpage
\lthtmlfigureA{align640}%
\begin{align}
\implies y_h&= e^{- \int^{}_{} p\left( x \right)  \operatorname{d}x }\cdot  c \notag
\end{align}%
\lthtmlfigureZ
\lthtmlcheckvsize\clearpage}

{\newpage\clearpage
\lthtmlinlinemathA{tex2html_wrap_inline1025}%
$y_1= e^{- \int^{}_{} p\left( x \right)  \operatorname{d}x }$%
\lthtmlindisplaymathZ
\lthtmlcheckvsize\clearpage}

{\newpage\clearpage
\lthtmlfigureA{align648}%
\begin{align}
  y_h&= y_1(x) \cdot  c
\end{align}%
\lthtmlfigureZ
\lthtmlcheckvsize\clearpage}

{\newpage\clearpage
\lthtmlinlinemathA{tex2html_wrap_inline1027}%
$p\left( x \right)$%
\lthtmlindisplaymathZ
\lthtmlcheckvsize\clearpage}

{\newpage\clearpage
\lthtmlinlinemathA{tex2html_wrap_inline1029}%
$c$%
\lthtmlindisplaymathZ
\lthtmlcheckvsize\clearpage}

{\newpage\clearpage
\lthtmlinlinemathA{tex2html_wrap_inline1031}%
$f \left( x \right) $%
\lthtmlindisplaymathZ
\lthtmlcheckvsize\clearpage}

{\newpage\clearpage
\lthtmlinlinemathA{tex2html_wrap_inline1033}%
$c = u\left( x \right)$%
\lthtmlindisplaymathZ
\lthtmlcheckvsize\clearpage}

{\newpage\clearpage
\lthtmlinlinemathA{tex2html_wrap_inline1035}%
$u\left( x \right)$%
\lthtmlindisplaymathZ
\lthtmlcheckvsize\clearpage}

{\newpage\clearpage
\lthtmlfigureA{align651}%
\begin{align}
y_p&= u\left( x \right)\times y_h\left( x \right)\notag \\
&= e^{- \int^{}_{} p\left( x \right)  \operatorname{d}x } \cdot  u\left( x \right)
\end{align}%
\lthtmlfigureZ
\lthtmlcheckvsize\clearpage}

{\newpage\clearpage
\lthtmlfigureA{align659}%
\begin{align}
  y_p &= e^{- \int^{}_{} p\left( x \right)  \operatorname{d}x } \cdot  u\left( x \right)\notag \\
\frac{\operatorname{d}y_p }{\operatorname{d} x}+ p\left( x \right)\cdot  &= f\left( x \right)\notag \\
\frac{\operatorname{d} }{\operatorname{d} x}\left( u\left( x \right)\cdot  y_1\left( x \right) \right)+ p\left( x \right)u\left( x \right)y_1\left( x\right)&=f \left( x \right) \notag \\
\frac{\operatorname{d}u }{\operatorname{d} x}\cdot  y_1\left( x \right)+ \frac{\operatorname{d}y_1 }{\operatorname{d} x}\cdot  u\left( x \right)+ p\left( x \right)\cdot  u\left( x \right)\cdot  y_1\left( x \right)&= f \left( x \right) \notag \\
u\left( x \right)\left( \frac{\operatorname{d}y_1 }{\operatorname{d} x} + p\left( x \right)y_1 \right)+ \frac{\operatorname{d}y }{\operatorname{d} x}\cdot  y_1\left( x \right)&= f\left( x \right)\notag \\
0 +  \frac{\operatorname{d}y }{\operatorname{d} x}- y_1\left( x \right)&= f \left( x \right) \notag \\
\frac{\operatorname{d}y }{\operatorname{d} x}&= f \left( x \right) /y_1\left( x \right)\notag \\
\int^{}_{} \frac{\operatorname{d}u }{\operatorname{d} x}  \operatorname{d}x &= \int^{}_{} f \left( x \right) /y_1\left( x \right)  \operatorname{d}x\notag \\
\int^{}_{}   \operatorname{d}u &= \int^{}_{} f \left( x \right) /y_1\left( x \right)  \operatorname{d}x\notag \\
u&= \int^{}_{} f \left( x \right) /y_1\left( x \right)  \operatorname{d}x
\end{align}%
\lthtmlfigureZ
\lthtmlcheckvsize\clearpage}

{\newpage\clearpage
\lthtmlfigureA{align703}%
\begin{align}
  u= \int^{}_{} f\left( x \right)\cdot  e^{\int^{}_{} p\left( x \right)  \operatorname{d}x }  \operatorname{d}x 
\end{align}%
\lthtmlfigureZ
\lthtmlcheckvsize\clearpage}

{\newpage\clearpage
\lthtmlinlinemathA{tex2html_wrap_inline1041}%
$y_p = u \cdot  y_1$%
\lthtmlindisplaymathZ
\lthtmlcheckvsize\clearpage}

{\newpage\clearpage
\lthtmlfigureA{align712}%
\begin{align}
  y_p&= \frac{1}{y_1}\cdot  \int^{}_{} f\left( x \right)\cdot  e^{\int^{}_{} p\left( x \right)  \operatorname{d}x }  \notag \\
  y_p&= e^{- \int^{}_{} p\left( x \right)  \operatorname{d}x } \int^{}_{} f\left( x \right)\cdot  e^{\int^{}_{} p\left( x \right)  \operatorname{d}x }  
\end{align}%
\lthtmlfigureZ
\lthtmlcheckvsize\clearpage}

{\newpage\clearpage
\lthtmlinlinemathA{tex2html_wrap_inline1043}%
$e^{\int^{}_{} p\left( x \right)  \operatorname{d}x }$%
\lthtmlindisplaymathZ
\lthtmlcheckvsize\clearpage}

{\newpage\clearpage
\lthtmlfigureA{align732}%
\begin{align}
% latex2html id marker 732

  e^{\int^{}_{} p\left( x \right)  \operatorname{d}x }\cdot  y_p&= e^{\int^{}_{} p\left( x \right)  \operatorname{d}x } \cdot  e^{- \int^{}_{} p\left( x \right)  \operatorname{d}x } \int^{}_{} f\left( x \right)\cdot  e^{\int^{}_{} p\left( x \right)  \operatorname{d}x }  \notag \\
  e^{\int^{}_{} p\left( x \right)  \operatorname{d}x }\cdot  y_p&=  \int^{}_{} f\left( x \right)\cdot  e^{\int^{}_{} p\left( x \right)  \operatorname{d}x } \notag \\
\frac{\operatorname{d} }{\operatorname{d} x}\left(   e^{\int^{}_{} p\left( x \right)  \operatorname{d}x }\cdot  y_p \right) &=  \frac{\operatorname{d} }{\operatorname{d} x}\left(  \int^{}_{} f\left( x \right)\cdot  e^{\int^{}_{} p\left( x \right)  \operatorname{d}x }  
 \right)\notag \\
 &= f\left( x \right)\cdot  e^{\int^{}_{} p\left( x \right)  \operatorname{d}x } \notag \\
 e^{\int^{}_{} p\left( x \right)  \operatorname{d}x } \frac{\operatorname{d}y }{\operatorname{d} x} + p\left( x \right)\cdot  e^{\int^{}_{} p\left( x \right)  \operatorname{d}x } \cdot  y &=  e^{\int^{}_{} p\left( x \right)  \operatorname{d}x } \cdot  f \left( x \right) \notag \\
 \implies  \frac{\operatorname{d}y }{\operatorname{d} x}   + p\left( x \right)\cdot  y &= f \left( x \right) \tag{\ref{linform}}
\end{align}%
\lthtmlfigureZ
\lthtmlcheckvsize\clearpage}

\stepcounter{subsection}
{\newpage\clearpage
\lthtmlfigureA{align795}%
\begin{align}
  \left( x+ 1 \right)\cdot  \frac{\operatorname{d}y }{\operatorname{d} x}+ y&= \ln{ \left( x \right) } \enspace ; \qquad y(1)= 10
\end{align}%
\lthtmlfigureZ
\lthtmlcheckvsize\clearpage}

{\newpage\clearpage
\lthtmlfigureA{align801}%
\begin{align}
  \frac{\operatorname{d}y }{\operatorname{d} x}+ \frac{y}{x+ 1} &=  \frac{\ln{ \left( x \right) }}{x+ 1} \enspace : \qquad \left( x\in \mathbb{R} \setminus \left\{-1,0\right\} \right)
\end{align}%
\lthtmlfigureZ
\lthtmlcheckvsize\clearpage}

{\newpage\clearpage
\lthtmlfigureA{align812}%
\begin{align}
  u&= e^{\int^{}_{} \frac{1}{x+ 1}  \operatorname{d}x }\notag \\
  &= e^{\int^{}_{} \ln{ \left| x+ 1 \right| }  \operatorname{d}x }\notag \\
  &= \left| x+ 1 \right|
\end{align}%
\lthtmlfigureZ
\lthtmlcheckvsize\clearpage}

{\newpage\clearpage
\lthtmlinlinemathA{tex2html_wrap_inline1049}%
$x > 0$%
\lthtmlindisplaymathZ
\lthtmlcheckvsize\clearpage}

{\newpage\clearpage
\lthtmlfigureA{align825}%
\begin{align}
\left( x+ 1 \right)\cdot  \frac{\operatorname{d}y }{\operatorname{d} x}+ y&= \ln{ \left( x \right) }\notag \\
 \implies  \frac{\operatorname{d} }{\operatorname{d} x}\left( \left( x+ 1 \right)\cdot  y \right)&= \ln{ \left( x \right) }\notag \\
\end{align}%
\lthtmlfigureZ
\lthtmlcheckvsize\clearpage}

{\newpage\clearpage
\lthtmlfigureA{align834}%
\begin{align}
 \int^{}_{} \frac{\operatorname{d} }{\operatorname{d} x}\left[ \left( x+ 1 \right)\cdot  y \right]  \operatorname{d}x&= \int^{}_{} \ln{ \left( x \right) }  \operatorname{d}x  \notag
\end{align}%
\lthtmlfigureZ
\lthtmlcheckvsize\clearpage}

{\newpage\clearpage
\lthtmlfigureA{align845}%
\begin{align}
\left( x+ 1 \right)\cdot  y&= \int^{}_{} \ln{ \left( x \right) }  \operatorname{d}x \notag\\
\end{align}%
\lthtmlfigureZ
\lthtmlcheckvsize\clearpage}

{\newpage\clearpage
\lthtmlfigureA{alignstar851}%
\begin{align*}
  \begin{matrix}
    u&= \ln{ \left( x \right) } && \operatorname{d} v&= \operatorname{d} x\\
    \operatorname{d} u&= \frac{1}{x}\operatorname{d} x &&v&= x
  \end{matrix}\\
\implies  \int^{}_{} u  \operatorname{d}v=  u\cdot  v + \int^{}_{} v  \operatorname{d}u 
\end{align*}%
\lthtmlfigureZ
\lthtmlcheckvsize\clearpage}

{\newpage\clearpage
\lthtmlfigureA{align868}%
\begin{align}
\left( x+ 1 \right)\cdot  y &= \ln{ \left( x \right)\cdot  x }- \int^{}_{}   \operatorname{d}x\notag \\
&= x\cdot  \left( \ln{ \left( x \right)- 1 } \right)+ c \notag \\
\ \implies  y&= \frac{x\cdot  \left( \ln{ \left( x \right)- 1 }+ c \right)}{x+ 1} \notag
\end{align}%
\lthtmlfigureZ
\lthtmlcheckvsize\clearpage}

{\newpage\clearpage
\lthtmlinlinemathA{tex2html_wrap_inline1051}%
$y(1)= 10$%
\lthtmlindisplaymathZ
\lthtmlcheckvsize\clearpage}

{\newpage\clearpage
\lthtmlfigureA{align878}%
\begin{align}
  10&= \frac{1\left( \ln{ \left( 1 \right)- 1 }+ c \right)}{2}\notag \\
  20&= 1\left( 0- 1 \right)+ c\notag \\
  c&= 19
\end{align}%
\lthtmlfigureZ
\lthtmlcheckvsize\clearpage}

{\newpage\clearpage
\lthtmldisplayA{displaymath1053}%
\begin{displaymath}
y= \frac{x\left( \ln{ \left( x \right)- 1 }+ 19 \right)}{x+ 1} \enspace ; \qquad \forall x \in \mathbb{C}\setminus\{-1,0\}
\end{displaymath}%
\lthtmldisplayZ
\lthtmlcheckvsize\clearpage}


\end{document}
